\documentclass[a4paper,12pt]{elsarticle_rfabbri}

%%%%%%%%%%%%%%%%%%%%%%%%%%%%%%%%%%%%%%%%%%%%%%%%%%
% prints all equation environments and nothing else, each cropped, one to a page.
%\usepackage[active,tightpage]{preview}
%\PreviewEnvironment{equation*}
%%%%%%%%%%%%%%%%%%%%%%%%%%%%%%%%%%%%%%%%%%%%%%%%%%
%\usepackage{xiiiemc}
\usepackage[small,bf]{caption}
\usepackage{graphicx}
\usepackage[usenames,dvipsnames]{color}
\usepackage{amsfonts}
\usepackage{amssymb}
\usepackage{amsmath}
\usepackage{indentfirst}
\usepackage{natbib}
\usepackage{fancyhdr}
%\usepackage[brazil]{babel}  				 					% CONVERSÃO EM PORTUGUÊS
\usepackage[utf8]{inputenc}                              	  % CONVERSÃO EM PORTUGUÊS
\usepackage{bbm} 								%para escrever double 1
\usepackage{algorithm}						% para escrever algoritmos
\usepackage{algpseudocode}				%para escrever pseudo-códigos
\usepackage{subfigure} 							%poe varias figuras como se fossem uma só
\usepackage{url}	
\usepackage[bookmarks,pdfborder={0 0 0}]{hyperref} % cria índice com referencias no arquivo pdf, 
\usepackage[top=2.0cm, bottom=3.3cm, left=2.5cm,right=2.5cm]{geometry} % Margens
\usepackage{setspace}    					     % para colocar espaçamento entre linhas
\definecolor{seagreen}{RGB}{46,139,87}
\usepackage{scalefnt}
\usepackage{listings}

\lstset{
  extendedchars=\true,
  inputencoding=utf8,
  language=Matlab,
  %showstringspaces=false,
  formfeed=\newpage,
  tabsize=4,
  %commentstyle=\itshape,
  basicstyle=\ttfamily\scriptsize,
  %basicstyle={\small\fontfamily{fvm}\fontseries{m}\selectfont},
  commentstyle=\color{Apricot}\bfseries,
  %commentstyle=\color{red}\itshape,
  stringstyle=\color{red},
  identifierstyle=\color{PineGreen},
  showstringspaces=false,
  keywordstyle=\color{blue}\bfseries,
  moredelim=[il][\large\textbf]{\#\# },
  morekeywords={models,range},
  numbers=left,
  numbersep=2pt,
  numberstyle=\tiny,%\color{blue}\bfseries,
  literate=%
  {ã}{{\~a}}1
  {â}{{\^a}}1
  {õ}{{\~o}}1
  {á}{{\'a}}1
  {ú}{{\'u}}1
  {í}{{\'i}}1
  {é}{{\'e}}1
  {Ç}{{\c{C}}}1
  {Õ}{{\~O}}1
  {Ê}{{\^E}}1
  {ó}{{\'o}}1
  {à}{{\`a}}1
  {Â}{{\^A}}1
  {ô}{{\^o}}1
  {ê}{{\^e}}1
  {ç}{{\c{c}}}1
}

%###########################################
%  outros comandos

\providecommand{\sin}{} \renewcommand{\sin}{\hspace{2pt}\mathrm{sen}}
\providecommand{\tan}{} \renewcommand{\tan}{\hspace{2pt}\mathrm{tg}}
\newcommand{\ie}{{\it i.e.}}
\newcommand{\etc}{{\it etc}}
\newcommand{\eg}{{\it e.g.}}
\newcommand{\keywordsname}{Keywords}
\newenvironment{keywords}{%
\noindent
        {\em \bfseries \keywordsname: }%
         \rm }
      {\endquotation \vskip 12bp}

\newcommand{\code}[2]{
 \vspace{1em}
 \subsubsection*{#1}
 \lstinputlisting{#2}
}


\begin{document}

\begin{frontmatter}

\title{Real-Time Perceptually-Tuned Color Detection} 



\author[iprj,aa]{Ricardo Fabbri\corref{corr1}}
\ead{rfabbri@iprj.uerj.br}
\ead[url]{www.lems.brown.edu/~rfabbri}


\author[aa]{Gilson Beck}

\author[ifsc,aa]{Vilson Vieira}
\ead{vilson@void.cc}

\author[icmc,aa]{Fernando Gorodscy}

\author[iprj]{Marcos Oliveira Couto Filho}

\author[ifsc,aa]{Renato Fabbri}

\address[iprj]{Instituto Polit\'{e}cnico, Universidade do Estado do Rio de
Janeiro\\C.P.: 97282 - 28601-970 - Nova Friburgo, RJ, Brazil}

\address[ifsc]{Instituto de F\'{i}sica de S\~{a}o Carlos (IFSC), Universidade de
S\~{a}o Paulo (USP)\\ Av.  Trabalhador S\~{a}o Carlense, 400,
13560-970 - S\~{a}o Carlos, SP, Brazil}

\address[icmc]{Instituto de Ci\^{e}ncias Matem\'{a}ticas e de
Computa\c{c}\~{a}o (ICMC), Universidade de
S\~{a}o Paulo (USP)\\ Av.  Trabalhador S\~{a}o Carlense, 400,
13560-970 - S\~{a}o Carlos, SP, Brazil}

\address[aa] {LabMacambira.sourceforge.net distributed hacker group}

\cortext[corr1]{Corresponding Author.\  Fax: +55 22 2533 5149}


%\thispagestyle{fancy}

\begin{abstract}
Color detector 
\end{abstract}

\begin{keyword}
Color detection\sep machine learning \sep interactive
computer vision \sep real time \sep photometric tracking \sep segmentation
\end{keyword}

\end{frontmatter}


\section{Introduction}

Color is attractive.

\section*{Acknowledgements}
We thank Teia, casa de criacao, who initially funded the development of the proposed
system.  The authors also thank the Brazilian agencies CNPq, CAPES, FAPESP, and
FAPERJ for later financial support.


\bibliographystyle{elsarticle-num}
\bibliography{personal}

\end{document}
\endinput
